\chapter{Conclusiones y trabajo futuro}
\label{chap:conclusiones}
\vspace{-0.2cm}

\lsection{Conclusiones}
Tras realizar el trabajo y probarlo con numerosas personas, concluyo que es un proyecto con una muy buena acogida, f�cil de desarrollar con las herramientas de las que disponemos y con mucho potencial, ya que incluye dos tecnolog�as que est�n en pleno desarrollo y que a�n tienen mucho que avanzar.

Otra ventaja es que hace uso de dispositivos de f�cil acceso por lo que no resultar�a demasiado caro de fabricar.

 

\lsection{Trabajo futuro}
Este trabajo es solo el inicio de lo que podr�a ser un proyecto muy interesante.

El robot que hemos construido es una extensi�n del sentido de la vista, pero se podr�a construir un robot controlado por una persona y que fuera la extensi�n de todos sus sentidos,permitiendo as� conocer y sentir el mundo sin necesidad de moverse de su casa.

Adem�s la realidad virtual es una herramienta muy potente, que no solo te permite modificar tu mundo a placer sino que te premita crear entornos totalmente distintos. Se podr�a estudiar c�mo interactuar con ese mundo virtual no solo a trav�s de la vista , sino con todos los sentidos.

Otro camino por el que se puede investigar es la uni�n de este proyecto con la dom�tica.
Creo que en un futuro no muy lejano, las casas ser�n casas domotizadas, que igual que se podr�n controlar desde el m�vil podremos controlass con las gafas de realidad virtual.





\newpage \thispagestyle{empty} % P�gina vac�a 