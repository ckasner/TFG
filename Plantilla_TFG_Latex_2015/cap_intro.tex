\chapter{Introducci�n} 
\label{chap:intro}

\vspace{-0.2cm}

\lsection{Motivaci�n del proyecto}
La motivaci�n de este proyecto es explorar las posibilidades que nos ofrece combinar la rob�tica con la realidad virtual.

El campo de la rob�tica lleva muchos a�os en desarrollo y mejora exponencialmente a lo largo del tiempo. Est� inmerso en nuestro d�a a d�a y no paramos de sorprendernos con nuevos avances como las impresoras 3D o los drones.

La realidad virtual es algo que, aunque lleva mucho tiempo estudiandose (En 1968 Ivan Sutherlan cre� el primer casco de realidad virtual), hasta hace bien poco, para la gran mayor�a de gente era solo ciencia ficci�n.

Ahora nos encontramos en un momento en el que ya podemos empezar a disrutar de la realidad virtual. Tenemos las CardBoard , las Oculus Rift... a nuestro alcance.
El uso m�s generalizado que se est� dando a estas herramientas es en el mundo de los videojuegos.

Este trabajo pretende aplicar estas dos tecnolog�as para hacer m�s f�cil el d�a a d�a de las personas, en concreto nos hemos centrado en las personas con discapacidad motora.

Hemos construido un robot que lleva incorporada una c�mara y que se comunica con las oculus rift.

El usuario puede controlar el movimiento del robto y de la c�mara solo con el movimiento de la cabeza, aunque no ser�a dif�cil adaptar este control para personas que les sea m�s f�cil controlarlo a trav�s de mandos.

El robot transmite a las oculus en tiempo real un video del entorno. Al ser las oculus gafas de realidad virtual, esto nos permite modificar el entorno seg�n las necesidades del usuario . P.ej: podr�amos controlar una casa domotizada.

De esta forma damos al usuario m�s independencia y le abrimos un mundo al que hasta ahora, ten�a dif�cil acceso.


Ejemplo de referencia a la bibliograf�a~\cite{article:Ejemplo}.

Ejemplo de imagen:
\begin{figure}[h]
  \centerline{
    \mbox{\includegraphics[width=3.00in]{images/logo_eps.eps}}
  }
  \caption{Ejemplo pie de figura 1}
  \label{fig:norm_Daugman}
\end{figure}

\lsection{Objetivos y enfoque}

\lsection{Metodolog�a y plan de trabajo}

Otro ejemplo de imagen:
\begin{figure}[h]
  \centerline{
    \mbox{\includegraphics[width=3.00in]{images/logo_uam.eps}}
  }
  \caption{Ejemplo pie de figura 2}
  \label{fig:norm_Daugman}
\end{figure}

\newpage \thispagestyle{empty} % P�gina vac�a 