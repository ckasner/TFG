\chapter{Realidad Virtual y rob�tica. Estado del arte}
\label{chap:estadodelarte}

\lsection{Introducci�n}

\lsection{Realidad virtual y salud}
\label{sec:VR y salud}

El uso de realidad virtual est� siendo muy �til en medicina, ya que permite estudiar comportamientos en diferentes situaciones de forma bastante realista.

Un ejemplo de esto es la evaluaci�n del deterioro cognitivo en personas que sufren esclerosis m�ltiple. Para dicha evaluaci�n se requiere una gran cantidad de datos sobre la velocidad de procesamiento de la informaci�n, atenci�n, etc...Esto se consegu�a haciendo m�ltiples test que no siempre eran fieles a la realidad.

Ahora se pueden crear entornos virtuales y estudiar el comportamiento de los usuarios en diversas situaciones de forma que el estudio es m�s realista y eficiente. [ref]


Tambi�n se est� estudiando el uso de programas de rehabilitaci�n para que los pacientes no necesiten trasladarse hasta el hospital.

Uno de los estudios se hizo a personas con hemiparesia, con un software que mostraba por pantalla el movimiento de los pies y les mandaba ejercicios para mejorar su capacidad motora. Los resultados de este estudio mostraron que la herramienta de realidad virtual era efectiva para la rehabilitaci�n y muy �til para los pacientes con dificultades para trasladarse hasta el hospital[ref]





\lsection{Rob�tica y realidad virtual.} 
\label{sec:robotica y VR}
La uni�n de rob�tica y realidad virtual es algo que se lleva haciendo desde hace bastante tiempo.

En el campo de la salud una de las aplicaciones son entrenamientos de cirug�ia. La realidad virtual permite crear un entorno semejante al quir�ofano. El estudiante hace uso de herramientas que simulan los utensilios quir�urgicos y puede hacer una "operaci�on virtual" sin riesgo para ning�un paciente y sin necesidad de gastar recursos o un equipo quir�urgico de apoyo . [ref]

En educaci�n hay varios programas de realidad virtual que hacen el aprendizaje m�s sencillo y completo a los estudiantes. Un ejemplo de esto es la visualizaci�n de brazos rob�ticos para dise�ar correctamente el movimiento.
Es relativamente complejo traducir cada movimiento de las articulaciones del brazo rob�tico en la posici�n finla de �ste. Si estas pruebas se hicieran directamente sobre el robot, ser�a muy sencillo romperlo. Con los programas de realidad virtual se simulan todos los movimientos sin peligro de malgastar recursos. [ref]

Otra aplicaci�n es el control remoto de robots, tanto con software que crea una interfaz virtual para que el usuario obtenga de forma clara toda la informaci�n del robot (bater�a, almacenamiento de datos, posici�n...) [ref NASA xD], como interfaces BCIs.

\lsection{BCIs, EEG y realidad virtual}
\label{sec:BCIs, EEG y realidad virtual}
Las interfaces BCIs (Brain Computer Interfaces) aspiran a que el usuario controle el robot a trav�s de pensamientos, al igual que controlamos nuestro cuerpo.Esta aplicaci�n sigue investig�ndose y se van consiguiendo grandes avances. [ref]??

 


\newpage \thispagestyle{empty} % P�gina vac�a 