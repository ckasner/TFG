\chapter*{Resumen}

\section*{Resumen}
La fusión de la Realidad Virtual con la robótica supone una apertura a infinitas posibilidades. 

Ambas tecnologías están en continuo desarrollo, de hecho la realidad virtual empezó a darse a conocer muy recientemente, a pesar de que su nacimiento data sobre  (año).

En los últimos años estamos viendo cómo la Realidad Virtual va haciéndose un hueco en las tecnologías que usamos habitualmente. Quizá el uso más comercial que se le está dando es en lo referente al mundo de los videojuegos.
Sin embargo se puede aplicar a muchos otros campos , en concreto en el campo de la salud , donde se están obteniendo muy buenos resultados (en cirugía, rehabilitación…).

Este trabajo pretende explorar el uso de la Realidad Virtual y la robótica como herramienta para ayudar a que personas con discapacidad motora sean más independientes.

Para conseguir este objetivo se ha construido un sistema que consta de: un robot y las Oculus Rift (gafas de realidad virtual).
El robot se controla desde las Oculus Rift, de forma que si el usuario lleva puestas las gafas, podrá controlar el robot con movimientos de cabeza.
El usuario verá a través de las gafas todo lo que vea el robot, actuando éste último como extensión de la vista del usuario.

Al utilizar las Oculus Rift dejamos abierta la posibilidad de ver, no solo el entorno en el que se mueve el robot, sino un entorno virtual creado por el propio usuario. Esta parte se deja como tema de estudio para trabajos futuros.

Además hemos diseñado el proyecto de forma que la conexión entre el robot y las gafas sea a través de red inalámbrica, lo que le da al robot una libertad de movimiento fundamental para el objetivo que se persigue.

A lo largo del desarrollo del trabajo han surgido varias complicaciones, la gran mayoría referentes a la utilización de los sensores de las Oculus y de la librería de las mismas, que aún no está perfectamente adaptada a la versión DK2 de las gafas.

A pesar de esto el proyecto se ha terminado con éxito, dejando abierta una línea de investigación y mejora sobre la que trabajar.


\section*{Palabras Clave}

\newpage

%-------------------------------------------------------------------------------------------------------------------------------------
\section*{Abstract}


\section*{Key words}
