\documentclass[10pt,notes,compress,usetitleprogressbar,aspectratio=1610]{beamer}

\usepackage[spanish, es-tabla]{babel}
\usepackage[T1]{fontenc}
\usepackage[utf8x]{inputenc}
\usepackage{tikztools}
\usepackage{pgfplots}
\usepackage{chronology}
\usepackage{tikz-3dplot}

\definecolor{palette1}{HTML}{1B9E77}
\definecolor{palette2}{HTML}{D95F02}
\definecolor{palette3}{HTML}{7570B3}
\definecolor{palette4}{HTML}{E7298A}
\definecolor{palette5}{HTML}{66A61E}
\definecolor{palette6}{HTML}{E6AB02}
\definecolor{palette7}{HTML}{A6761D}
\definecolor{palette8}{HTML}{666666}

\usetheme{metropolis}

\metroset{
	color/theme = eps,
	fonttheme = palatino,
	titleformat = allsmallcaps,
	titleformat subtitle = regular,
	numbering = fraction,
	progressbar = frametitle,
	titleformat plain = regular,
	logo = eps,
	section as frame subtitle
}

\renewcommand{\event}[3][e]{
	\pgfmathsetlength\xstop{(#2-\theyearstart)*\unit}%
	\draw[fill=black,draw=none,opacity=0.5]%
		(\xstop, 0) circle (.2\unit)%
		node[opacity=1,rotate=45,right=0\unit, yshift = 0.3cm] {#3};
}

% For input of gnuplot files. Automatically resizes the image based on linewidth.
\newcommand{\inputgnuplot}[2][1]{
	\vspace{-2ex}
	\resizebox{#1\linewidth}{!}{
		\IfFileExists{#2.tex}{
			\fontsize{7pt}{7pt}\fontfamily{ppl}\selectfont
			\input{#2.tex}
			\vspace{-15pt} % Adjust for font size
		}{
			\PackageWarning{epstfg}{File `#2.tex' not found.}
			\fbox{Gnuplot not found.\begin{minipage}{4.7in}\hfill\vspace{3in}\end{minipage}}
		}
	}
	\vspace{-3ex}
}


\title{Monitorizaci\'on, captura y almacenamiento inteligente de tr\'afico de red a 40 Gbps}
\subtitle{Trabajo Fin de Grado | Doble Grado en Ingenier\'ia Inform\'atica - Matem\'aticas}
\author{Guillermo Juli\'an Moreno}
\date{Junio 2016}

\institute{HPCN \\ Dept. Tecnología Electr\'onica y de las comunicaciones}


\setbeameroption{show notes}
% \setbeameroption{show only notes}

\begin{document}

\maketitle

\begin{frame}{Tabla de contenidos}
  \setbeamertemplate{section in toc}[sections numbered]
  \tableofcontents[hideallsubsections]
\end{frame}

\section{Introducción y motivación}

\begin{frame}
\frametitle{Redes Ethernet}

\begin{figure}[t]
\centering
\begin{chronology}{1999}{2016}{0.8\textwidth}
\event{2002}{\footnotesize 1ª versión 10 GbE}
\event{2006}{\footnotesize Estándar 10 GbE}
\event{2010}{\footnotesize Estándar 40/100 GbE}
\event{2014}{\footnotesize Intel XL710 40 GbE}
\event{2011}{\footnotesize Mellanox ConnectX-3 40 GbE}
\end{chronology}
\caption{Cronología de redes Ethernet de alta velocidad}
\end{figure}

Cada vez más CPDs aumentan la velocidad de sus redes Ethernet: las soluciones a 40~Gbps ya empiezan a desplegarse comercialmente.

Las \textbf{necesidades de monitorización} persisten: detección de problemas y anomalías, diagnósticos de red, detección de intrusiones...

\note{Para la monitorización se usan soluciones específicas: la pila de red de Linux es demasiado genérica y realiza tareas adicionales que no hacen falta si sólo buscamos captura de tráfico.

Son especialmente interesantes las que usan hardware y servidores no especializados por su bajo coste. Una de esas soluciones es HPCAP.}
\end{frame}

\begin{frame}
\frametitle{HPCAP}

HPCAP, tesis doctoral de Víctor Moreno: \textit{driver} Linux para captura y almacenamiento de tráfico a 10 Gbps en tarjetas Intel, con uso bajo de recursos.
\vspace{3em}
\begin{alertblock}{Objetivo del TFG}
Ampliación y mejora de HPCAP para la recepción a 40 Gbps.
\end{alertblock}
\end{frame}

\begin{frame}
\frametitle{Recepción 40G: Retos}

\begin{columns}
\begin{column}{0.5\textwidth}
\begin{itemize}
\item Nos acercamos a los límites de velocidad del procesador. \note{Lo que en entornos normales ni siquiera se considera, como las latencias de acceso a la caché, a estas velocidades pueden significar una parte importante del tiempo disponible para procesar cada paquete. Incluso un fallo del predictor de ramas puede suponer un cuarto del tiempo disponible.}
\item Tampoco hay mucho margen a nivel PCI: el ancho de banda máximo de PCIe 3.0 x8 es de unos 55 Gbps.
\item Se genera una gran cantidad de datos: 420 TB al día en un enlace 40 Gbps. \note{Si queremos almacenar estos datos vamos a tener que reducirlos de alguna forma, incluso antes de que lleguen a las aplicaciones de análisis que lean del driver.}
\end{itemize}
\end{column}
\begin{column}{0.5\textwidth}
\begin{figure}
\centering
\begin{tikzpicture}
\begin{axis}[
	width = 0.9\textwidth,
	horizontal mbarplot, xlabel = {\footnotesize Ciclos},
	symbolic y coords = {Cach\'e L3,Cach\'e L2,Cach\'e L1,Fallo \textit{branch}},
	ytick = data,
	x tick label style={font=\footnotesize},
    y tick label style={font=\footnotesize},
    xmax = 75,
    xtick = {0, 10, 20, 30, 40, 50, 57}
    ]
	\addplot plot coordinates {
		(4,Cach\'e L1)
		(12,Cach\'e L2)
		(44,Cach\'e L3)
		(16,Fallo \textit{branch})
	};
	\draw[mDarkBrown, dashed, thick] ({rel axis cs:{58/75},0}) -- ({rel axis cs:{58/75},1}) node[above, yshift = 0.05cm, midway, sloped, align = center, rotate = 180] {\footnotesize Tiempo entre  paquetes (64b)};
\end{axis}
\end{tikzpicture}
%\caption{Latencias en CPUs Intel Skylake a 3,4 GHz}
\end{figure}
\end{column}
\end{columns}
\end{frame}

\section{Desarrollo e implementación}

\begin{frame}{Arquitectura previa HPCAP}

\begin{figure}[hbtp]
\centering
\begin{tikzpicture}[scale = 0.7]
\begin{scope}[xshift = -6cm, yshift = -0.5cm]
\draw[fill = green!60!black, fill opacity = 0.3] (-0.1, 1.2) -- (0, 1.2) -- (0,-0.3) -- (0,0) -- (0.7,0) -- (0.7,-0.15) -- (1.7, -0.15) -- (1.7, 0) -- (1.8, 0) -- (1.8, -0.15) -- (2, -0.15) -- (2,0) -- (2.3, 0) -- (2.3, 1) -- (0,1) -- (0, 1.2) -- cycle ;
\end{scope}

\draw[->] (-3.6, 0) -- node[above, midway] {\small DMA} (-1.1, 0);

\fill[gray!20!white] (0,0) -- (0,1) arc (90:-30:1cm) -- cycle;
\foreach \x in {0, 30, ..., 360}
	\draw ({cos(\x)}, {sin(\x)}) -- ({-cos(\x)}, {-sin(\x)});
\draw (0,0) circle [radius = 1cm];
\draw[black, fill = white] (0,0) circle [radius = 0.5cm]; % Poor man's \clip for intersecting regions

\node[vnlin, label = {above:{\footnotesize tail}}] at (0,1.15) {};

\node[hnlin, label = {below right:{\footnotesize head}}, rotate = -30] at ({1.15 * cos(-30)}, {1.15 * sin(-30)}) {};

\node at (0, -1.8) {\small Anillo de descriptores};
\node at (8, -1.8) {\small Buffer de datos};
\node at (-4.85, -1.8) {\small NIC};

\fill[gray!20!white] (4, 0) rectangle (9, 0.5);
\draw (4, 0) rectangle (12,0.5);

\draw[thick] (5, 0) -- (5, 0.5);
\draw[thick] (5.5, 0) -- (5.5, 0.5);
\draw[thick] (7, 0) -- (7, 0.5);
\draw[thick] (9, 0) -- (9, 0.5);

\draw[->] ({1.05 * cos(75)}, {1.05 * sin(75)}) to [bend left] (4.2, 0.55);
\draw[->] ({1.05 * cos(45)}, {1.05 * sin(45)}) to [bend left] (5.2, 0.55);
\draw[->] ({1.05 * cos(15)}, {1.05 * sin(15)}) to [bend right] (5.7, -0.05);
\draw[->] ({1.05 * cos(-15)}, {1.05 * sin(-15)}) to [bend right] (7.2, -0.05);

\end{tikzpicture}

% \caption{Anillo de descriptores}
\end{figure}

\begin{itemize}
\item Un único hilo de copia al \textit{buffer} de datos.
\item Uso de \textit{padding} para poder almacenar por bloques. \note{El almacenamiento por bloques es más eficiente que paquete a paquete.}
\item Los descriptores se devuelven de manera ordenada a la tarjeta.
\end{itemize}

\end{frame}

\begin{frame}
\frametitle{Requisitos}

\note{En un primer momento podríamos decir que este problema ya está resuelto: ya existen colas concurrentes, ¿no? No, si pretendemos que vaya a 40 Gbps.}

\begin{itemize}
\item Arquitectura multihilo.
\item Semáforos, \textit{mutex} o \textit{spinlocks} son demasiado lentos.
\item Minimizar los puntos de sincronización para evitar fallos de caché y latencia por sincronización entre núcleos.
\item Mantener al máximo el orden de recepción.
\item Marcado de tiempo preciso.
\end{itemize}
\end{frame}

\begin{frame}
\frametitle{Solución}

Aprovechamos las características de HPCAP para diseñar una arquitectura eficiente.

\begin{itemize}
\item \textbf{Anillo de recepción fijo}: Asignamos un hilo a cada segmento del anillo de recepción. Así evitamos sincronizar la lectura, sólo sincronizamos devolución en el borde de cada segmento con una variable atómica. \note[item]{La variable atómica marca si el segmento anterior ha acabado.}
\item \textbf{Buffers de tamaño prefijado}: Mantenemos una variable atómica con la posición de escritura, dejamos que el \textit{overflow} de C haga el módulo implícitamente. \note[item]{Habría que reiniciar la posición a 0 cuando llegásemos al final del \textit{buffer}, lo que implicaría más sincronización. Pero si el tamaño del buffer es siempre potencia de 2, no tenemos que preocuparnos. Esto además nos permite controlar el padding al inicio y final del fichero.}
\item \textbf{Lectores en espacio de usuario e hilos funcionando sin paradas}: No es necesario marcar cuándo se acaba la copia, ya que siempre acaba antes de que el lector sepa que hay nuevos datos.
\end{itemize}

Por último, se aprovecha el marcado de tiempo hardware para evitar problemas de precisión.

\end{frame}

\begin{frame}{Almacenamiento inteligente}

\begin{columns}
\begin{column}{0.5\textwidth}
\alert{Problema} 40 Gbps son muchos datos a procesar y almacenar ($\sim$400 TB al día).

Soluciones a nivel de hilos de recepción:
\begin{itemize}
\item \textbf{Filtrado}: Configurados en vivo por la aplicación lectora sin parar la captura en ningún momento.
\item \textbf{Almacenamiento selectivo}: Selección de los rangos de bytes que se guardan en el \textit{buffer}.
\end{itemize}
\end{column}
\begin{column}{0.5\textwidth}
\begin{figure}[btp]
\inputgnuplot{gnuplot/caplen-effects}
\caption{Tasa efectiva limitando el tamaño de paquete}
\label{fig:Desarrollo:CaplenEffects}
\end{figure}
\end{column}
\end{columns}
\end{frame}

\begin{frame}{Sistemas multiprocesador}

\begin{figure}[tbp]
\centering
\footnotesize
\resizebox{0.7\textwidth}{!}{

\begin{tikzpicture}
\begin{scope}
\node[palette1, anchor = east] at (6, -2.3) {Nodo NUMA 0};
\draw[palette1, fill = palette1, fill opacity = 0.1] (0, -2) rectangle (6,2);
\draw[fill = white] (5.5, 1.2) rectangle (3.9, -1.2);
\draw (4.7, 1.2) -- (4.7, -1.2);
\draw (5.5, 0.4) -- (3.9, 0.4);
\draw (5.5, -0.4) -- (3.9, -0.4);

\foreach[count = \iy] \y in {0.8, 0, -0.8} {
	\foreach[count = \ix] \x in {4.3, 5.1} {
		\pgfmathsetmacro{\nodename}{2 * (\iy - 1) + \ix - 1}
		\node at (\x, \y) {\pgfmathprintnumber[int trunc]{\nodename}};
	}
}

\node at (4.7, 1.43) {CPU 0};

\draw[palette1, fill = palette1, fill opacity = 0.4] (0.2, 1.8) rectangle (1.8, -1.8);
\node[align = center] at (1, 0) {Memoria \\ local \\ nodo 0};

\fill[palette1, pattern = north east lines] (1.8, -0.6) rectangle (3.9, 0.6);
\node at (2.85, -0.8) {Bus local};

\node[draw, rectangle, minimum width = 2cm, minimum height = 1cm] (NIC) at (2.85, 3) {NIC};
\draw[very thick] (2.85, 0.6) -- (NIC);
\end{scope}

\begin{scope}[xshift = 14cm, xscale = -1]
\node[palette2, anchor = west] at (6, -2.3) {Nodo NUMA 1};
\draw[palette2, fill = palette2, fill opacity = 0.1] (0, -2) rectangle (6,2);
\draw[fill = white] (5.5, 1.2) rectangle (3.9, -1.2);
\draw (4.7, 1.2) -- (4.7, -1.2);
\draw (5.5, 0.4) -- (3.9, 0.4);
\draw (5.5, -0.4) -- (3.9, -0.4);

\foreach[count = \iy] \y in {0.8, 0, -0.8} {
	\foreach[count = \ix] \x in {5.1, 4.3} {
		\pgfmathsetmacro{\nodename}{2 * (\iy - 1) + \ix - 1}
		\node at (\x, \y) {\pgfmathprintnumber[int trunc]{\nodename}};
	}
}

\node at (4.7, 1.43) {CPU 1};

\draw[palette2, fill = palette2, fill opacity = 0.4] (0.2, 1.8) rectangle (1.8, -1.8);
\node[align = center] at (1, 0) {Memoria \\ local \\ nodo 1};

\fill[palette2, pattern = north west lines] (1.8, -0.6) rectangle (3.9, 0.6);
\node at (2.85, -0.8) {Bus local};


\node[draw, rectangle, minimum width = 2cm, minimum height = 1cm] (RAID) at (2.85, 3) {RAID};
\draw[very thick] (2.85, 0.6) -- (RAID);
\end{scope}

\fill[gray, opacity = 0.2] (5.5, 0.3) rectangle (8.5, -0.3);
\fill[pattern = horizontal lines, pattern color = gray] (5.5, 0.3) rectangle (8.5, -0.3);
\node at (7, -0.45) {\small Interconexi\'on};
\end{tikzpicture}
}
\caption{Arquitectura NUMA de los sistemas de pruebas.}
\label{fig:Validacion:NUMAArch}
\end{figure}

Para maximizar el rendimiento, se usan sistemas multiprocesador y se tiene en cuenta su arquitectura NUMA. Los hilos de recepción se ejecutan en el nodo con la NIC, los de almacenamiento en el nodo con el RAID.

\end{frame}

\section{Validación experimental}

\end{document}
